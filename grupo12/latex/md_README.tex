2º Projeto de Introdução à Computação 2018/2019

\char`\"{}\+Settlers of Catan\char`\"{} Simples \begin{DoxyVerb}Trabalho realizado por: 
Miguel Fernández - a21803644
Nuno Galego - a21804797
João Rebelo - a21805230
\end{DoxyVerb}


Descrição\+: Neste ficheiro está a descrição do jogo \char`\"{}\+Catan Simples\char`\"{}, as regras do jogo com dados importantes da sua execução e conclusões que tirámos com tudo isto. No ficheiro do projeto estão também 2 fluxogramas do programa tal como todos os ficheiros .c e headers necessários para o funcionamento do mesmo.

Estruturas de dados feitas\+: -\/\+Estrutura de mapa; -\/\+Estrutura de jogador que inclui uma estrutura de recursos e território; -\/\+Estrutura de Nó para ser utilizada na criação da grelha chamando a estrutura do mapa;

Algoritmos relevantes\+: -\/\+Header criado para alocar a memória para o mapa; -\/\+Header de jogador que inclui todos os protótipos de função para o funcionamento de 1 turno;

Manual de utilizador\+: Para compilar é só correr o ficheiro makefile. Como jogar\+: Jogo 1v1, ganha o jogador que chegar mais rápido aos 6 Pontos de Vitória. O jogo inicia-\/se, cada jogador tem que escrever o seu nome e escolher a sua casa de partida. De seguida o Jogador 1 lança os dados, se o número que este lançou for igual a uma das coordenadas, Norte, Sul, Este e Oeste, o Jogador 1 ganha o recurso que estiver na coordenada que acertou, se o número que lançar não for igual a nenhuma coordenada, este não ganha nada. Depois passa para a próxima fase em que o jogador pode trocar 4 de um recurso por 1 de outro recurso ou trocar 10 de um recurso por 1 ponto de vitória. Pode também comprar mais aldeias com 1 de cada recurso menos ferro e/outransformar aldeias em cidades (cada aldeia comprada arrecada 1 ponto de vitória para o comprador e cada cidade arrecada 2 pontos de vitória). De seguida joga o Jogador 2 tendo este as mesmas escolhas, o jogo só acaba quando 1 dos jogadores ganhar.

Conclusões e matéria aprendida\+: Tivemos problemas com arrays, listas e matrizes na criação da grelha, inclusive ao colocar a grelha em forma de quadro colocando-\/a horizontalmente. Foi difícil criar as estruturas e garantir que os valores destas fossem alterados conforme o desejado. Conseguimos organizar programas em bibliotecas sem ter necessidade de definir variáveis globaisevitando ficheiros .c demasiado extensos e desorganizados. Apesar de termos conseguido organizar, devido a problemas com o upload dos ficheiros para o repositório infelizmente aconteceu algum tipo de conflito entre ficheiros revertendo a versão quase completa do programa para um estado bem mais antigo. Com isso houve bastantes erros que começaram a aparecer que não conseguimos mesmo resolver, fazendo com que não conseguissemos que o I\+NI fosse lido corretamente, por exemplo. Como este problema aconteceu um dia antes da entrega, para evitar não entregar um projeto decidimos criar a versão catan\+Ultimo\+Recurso que não lê o I\+NI nem usa tantos headers como a versão que falhou. Incluimos ambas as versões na diretoria do grupo12.

Referências\+: Para este projeto, as referências mais diretas que utilizámos foram tiradas da aula 12 de IC, utilizando os cprogs e os slides dados pelo professor que nos ajudaram imenso a organizar e a estruturar o nosso programa. Também tirámos algumas dúvidas diretamente com o professor, para outras dúvidas usámos um livro chamado \char`\"{}\+Linguagem C\char`\"{} de Luís Damasou então pesquisámos no Stack Overflow. Tivemos algumas discussões de ideias com colegas, o João David e o Rodrigo Pinheiro.

 